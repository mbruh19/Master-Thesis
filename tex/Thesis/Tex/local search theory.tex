\section{Local Search Model for Binary and Ternary Neural Networks}

A local search (LS) algorithm starts from a solution, which may be a random assignment of values to the decision variables. From here it moves from the current solution to a neighboring solution in the hope of improving a function $f$. A solution is denoted by $s$, and the set of neighboring solutions of $s$, $N(s)$, is called the neighborhood of $s$. In the following I will define the LS model I have used to train BNNs and TNNs and explain how local search is used. 

\subsection{Notation}
A BNN is a network where the weights and activations are restricted to $\pm 1$, whereas allowing the weights to be equal to 0 gives a Ternary Neural Network (TNN). This model applies for fully connected NNs, where the input comes from the data. I denote by $N = \{N_0, N_1, \ldots, N_L\}$ the set of layers in the network where $N_0$ is the input layer and $N_L$ is the last layer. For each layer, the set of neurons are denoted by $N_l = \{1, 2, \ldots, n_l\}$ such that the width of layer $l$ is $n_l$. The decision variables are the weights between each layer. The weight between neuron $u$ in layer $N_{l-1}$ and neuron $v$ in layer $N_l$ is denoted by $w_{uv}^l$. For the hidden layers, the activation function used to binarize the output is given by: 

\begin{align}
    \label{act} p(x) = 2 \cdot \mathbb{I} (x \geq 0) - 1
\end{align}

\noindent The training set can be written as $TR = \{ (\mathbf{x}^1, y^1), \ldots , (\mathbf{x}^T, y^T) \}$ such that $\mathbf{x}^i \in \mathbb{R}^{n_0}$ and $y^i$ is the label of the instance $i$ for every $i\in \{1, 2, \ldots, T\}$. It is beneficial to keep track of the preactivation values for all training instances for all the neurons in the layers $\{N_1, \ldots, N_L\}$. I use $s_v^{il}$ to denote the preactivation for instance $i$ at neuron $v$ in layer $l$. Similarly, for all the hidden layers $\{N_1, \ldots, N_{l-1}\}$, the output of the activation is denoted by $u_v^{il} = p(s_v ^{il})$. 

\noindent All of the above can also be written in terms of matrices and vectors. The input matrix is then denoted by $X \in \mathbb{R}^{T \times n_0}$ such that every row corresponds to the input of a specific instance and $Y$ is a vector with the labels. Then the mathematical model for a BNN can be written as:

\begin{align}
    \label{obj} \max \;\quad  & f(S^L, Y) \\
    \label{c1} \mbox{s.t.}\quad  & U^0 = X \\
    \label{c2} & S^l = U^{l-1}W_l \quad \quad \quad \quad \quad \forall l \in \{1, \ldots, L\} \\
    \label{c3} & U^l = p(S^l) \quad \quad \quad \quad \quad \quad \forall l \in \{1, \ldots, L- 1\} \\
    \label{c4} & W^l \in \{-1, 1\}^{n_{l-1} \times n_l} \quad \quad \forall l \in \{1, \ldots, L\} \\
    \label{c5} & X \in \mathbb{R}^{T \times n_0}
\end{align}

\noindent where (\ref{act}) is applied elementwise. Notice that I maximize a function, which is in contrast to standard ML, where the crossentropy loss is often minimized. I overcome this by multiplying by $-1$, whenever I use a loss function that is tradtionally minimized. If the input is real-valued, then $S_1$ is also real-valued but because of the binary activations and weights, $S^l$ is integer-valued for $l\in \{2, \ldots, L\}$. For a TNN, 0 is included to be an option for the $W$-matrices. Occassionally I will also use the notation $S^l_v$, which is a vector of the preactivation values in layer $l$, but only for neuron $v$. 

\subsection{Solution Generation, Neighborhood and Move}

As earlier mentioned, the decision variables are the weights between the layers. The preactivations and activations are determined by the input data and the weights. The number of variables in the model are thus given by $k = n_0\cdot n_1 + n_1 \cdot n_2 + \ldots + n_{L-1} \cdot n_L$. This means that the number of possible solutions for a TNN is $3^k$ and for a BNN it is $2^k$. To generate a solution, I use a uniform, random assignment of the allowed values to the variables. I use a 1-exchange neighborhood such that the neighbors to a solution $s$ are the solutions $s'$ in which it is only a single weight that has another value. This gives a neighborhood of size $k$ for a BNN and $2k$ for a TNN. A move is thus defined as the operation that take a solution $s$ to a new solution $s'$ by changing the value of exactly one weight. For a BNN this is simply a 'flip' from -1 to 1 or from 1 to -1, whereas for a TNN there is always two new values that a weight can take. \\

Often the number of weights in a NN are very large, and even though I only work with relative small NNs the number of weights quickly grows large. As a result, I found it to be inefficient to implement a function that tests the effect of a single move. For this reason I developed another approach, in which I evaluate several moves at once. I will elaborate on this later with an example. 

\subsection{Objective Functions}
As discussed earlier, when training NNs surrogate loss functions are often used, because the performance measure one is often interested in is non-differentiable, non-continuous and it might not be flexible enough. From now on I will denote the function to be maximized as an objective function, and whenever this corresponds to a loss function, which are normally minimized, I simply multiply the objective function value by -1 to get a maximization problem. In ML the objective function is used to compute gradients, that can be used to update the parameters in a gradient descent algorithm. In a LS context, an objective function does not necessarily have to be differentiable. In my implementation I support three different type of objective functions.

\subsubsection{Cross Entropy Objective Function}
I have already introduced the cross entropy function earlier. As I have implemented everything as a maximization problem, the objective function becomes:

\begin{align}
    \label{cs} \max \sum_{i=1}^m \log P(\mathbf{y}^{(i)} \mid \mathbf{x}^{(i)})
\end{align}
where $m$ is the number of samples in the batch, $\mathbf{y}^{(i)}$ is the label of sample $i$ with input data $\mathbf{x}^{(i)}$. Since the probabilities are always between 0 and 1, and the logarithm takes each probability and map them to a value between $(-\infty, 0)$, the range of this objective function is $(-\infty, 0)$. 

\subsubsection{Integer Max Margin Objectice Function}
The cross entropy objective function work with real-valued numbers. It might be beneficial to work with an integer-valued objective function for memory effiency reasons. As such I propose what I call the Integer Max Margin objective function. It builds on exactly the same intuition as the others, which for the multiclassification problem is to maximize the preactivation value for the neuron corresponding to the correct label. As earlier let $i$ denote a training instance and let $y^i$ be the true label, and let $\hat{y}^i$ be the predicted value defined as:

\begin{align}
    \label{y_hat} \hat{y}^i = \max _{v\in \{1, \ldots, n_L \} } s_v^{iL}
\end{align}

\noindent The goal is to stay within integers and at the same time predict confidently. One way to do this is to maximize the distance between the true label and the label with highest preactivation value, corresponding to highest probability, that are not the true label. So if $\hat{y}^i \neq y^i$ then $s_{\hat{y}^i}^{iL} > s_{y^i}^{iL}$ and the contribution to the objective function in this case will be $s^{iL}_{y^i} - s_{\hat{y}^i}^{iL}$ which is negative. The first goal would then be to minimize the gap between the wrongly prediction and the true label. On the other hand, if $\hat{y}^i = y^i$, then the model predicts correctly and in this case the goal is to maximize the distance to the label that come closest. This term can be written as: $s_{y^i}^{iL} - \max_{v \in \{1, \ldots, n_L\} \setminus y^i } s_v ^{iL}$. This can be summarized into a single line, so that the objective function, for a single instance, is given by: 

\begin{align}
    \label{integer_objective} f(\mathbf{x}^i, y^i) = s_{y ^i} ^{iL} - \max_{v \in \{1, \ldots, n_L\} \setminus y^i } s_v ^{iL}
\end{align}
As a result, the objective function for all $m$ samples in a batch becomes:
\begin{align}
    \label{int} \max \sum_{i=1} ^m \big[ s_{y ^i} ^{iL} - \max_{v \in \{1, \ldots, n_L\} \setminus y^i } s_v ^{iL}  \big]
\end{align}

The range of this objective function is theoretically given by $[-m \cdot n_{L-1}, m \cdot n_{L-1}]$. 

\noindent For the binary classification problem, this does not work as there is only one neuron at the final layer. The bounds for the preactivation values are determined by the number of neurons in the second last layer, $n_{L-1}$, such that the bounds are given by $[-n_{L-1}, n_{L-1}]$. For binary classification problems, the labels can also be encoded as -1 and 1. For the samples labeled 1, the goal is to get the preactivation values as close to $n_{L-1}$ as possible, or equivalently, maximize the distance to $-n_{L-1}$. The goal is opposite for the samples labeled -1, here the goal is to get the preactivation values as close to $-n_{L-1}$ as possible. For a single instance, this can be summarized into:
\begin{align}
    \label{integer_objective_binary} f(\mathbf{x}, y) = y\cdot s^L + n_{L-1}
\end{align}
To see this, take the case of $y=1$, then we want to maximize $s^L - (-n_{L-1})$ which is equivalent to (\ref{integer_objective_binary}). On the other hand, if $y = -1$, then we want to maximize the distance from $s^L$ to $n_{L-1}$, which is the same as $n_{L-1} - s^L$. Again, when multiplying $s^L$ with $y$ this is equivalent to (\ref{integer_objective_binary}). For all $m$ samples, the objective function is thus given by:
\begin{align}
    \label{int_binary} \max \sum_{i=1} ^m \big[ y^i \cdot S^{iL} + n_{L-1} \big]
\end{align}

\subsubsection{Comparison Between Cross Entropy and Integer Objective Function}
To illustrate the difference between this objective function and the crossentropy, suppose we are in a setting with 5 different classes, the correct label is the first label and the preactivation values for the 5 neurons in the output layer is given by $[4, 2, -2, -2, -2]$, such that the contribution to the crossentropy objective function is $-0.133456$ (after multiplying by -1). The contribution to the integer objective function is 2. Clearly, we are most interested in decreasing the preactivation value for the neuron with value 2, as this would mean a more confident (and correct) prediction of our instance. However, suppose we find a move that decreases the preactivation value of the third neuron from -2 to -4. This gives a new contribution from the crossentropy function equal to $-0.131558$, which is only a very small improvement. On the other hand, this is not an improvement for the integer objective function, where an improvement only is found if either the preactivation value of the first neuron increases or the preactivation value for the second neuron decreases. 

\subsubsection{Brier Score}
So far I have introduced two objective functions, whose range is quite large. As I, later on, want to test whether it is possible to add a regularization parameter to the objective function to minimize the number of connections in a TNN, it is desirable to also have an objective function, which have a more constrained range. The Brier score, which measures the accuracy of probabilistic predictions, gives this possibility. For binary classification using 0-1 encoding of the labels, the Brier score is defined as: 

\begin{align}
    BS = \frac{1}{m} \sum_{i = 1} ^m (p(y^i) - y^i) ^2 
\end{align}
where $p(y^i)$ is the probability that $y^i$ is 1 and $y^i$ is the actual outcome. This corresponds to the mean squared error, and it gives a value between 0 and 1, where 0 is the best score achievable. I formulate it without taking the mean and convert it to a maximization problem by multiplying with $-1$, so the objective function for binary classification becomes:
\begin{align}
    \label{BS_binary} \max \sum_{i=1}^m - (p(y^i) - y^i) ^2 
\end{align}
The range of this objective function value is $(-m, 0)$. \\
\noindent For the multiclassification task, the original BS score is defined by: 
\begin{align}
    BS = \frac{1} {m} \sum_{i=1} ^m \sum_{t=1} ^C (p(y^{ti}) - y^{ti})
\end{align}
where $C$ is the number of classes, $p(y^{ti})$ is the predicted probability for class $t$ for instance $i$. $y^{ti}$ is 1 of instance $i$ belongs to the $t$-th class and zero otherwise. Here the range is double, from zero to two. In my implementation I use:
\begin{align}
    \label{BS} \max \sum_{i=1}^m \sum_{t=1} ^C - (p(y^{ti}) - y^{ti})
\end{align}
The range of this objective function is thus $(-2m, 0)$. 
\subsubsection{Minimizing the number of connections}
In a TNN it is possible to add a second term to the objective function measuring how many active connections are in the model, i.e. how many weights are not zero. The idea is, that the model should be as simple as possible to avoid overfitting. In this context simple means having fewer active connections. Thus, the goal of the second objective term is to minimize the number of connections. As I am working with maximization, the second objective term would look like this:

\begin{align}
    \label{min_connections}  - \alpha \sum_{l\in{1, \ldots, L}} \sum_{u \in N_{l-1}} \sum_{v \in N_l}  \text{abs}(w^l_{uv})
\end{align}

\noindent where $\alpha$ is a weight used to penalize the number of connections. The term is negative as it is put into an maximization problem. The choice of $\alpha$ depends on the objective function it is used in, as the weighting of this second term should depend on what values the first term is taking. As such the Brier score objective function is an ideal choice to use in combination with this term, as the value of that is constrained. As an example suppose the number of training samples is given by $T = 1000$ and the number of weights in the neural network is 10000. Assume it is a multiclassification task. Then the objective function value for the Brier score is in the interval $(-2000, 0)$ and the second term would be in the interval $[-10000, 0]$, with $\alpha = 1$ that is. The total objective function value is as a result in the interval $(-12000, 0)$. A too high value of $\alpha$ would simply put all weight values to zero, but at the cost of predicting correctly. But a small value of $\alpha$, say $0.001$ would make the second term to be in the interval $[-10, 0]$, and it would no longer be beneficial to set all weights equal to zero at the cost of predicting correctly. 

\subsection{Delta Evaluation}
A very important part of a LS algorithm is how to select moves to commit to a solution. Often this is done by selecting a possible move, evaluating that move and get its delta value, which is a value telling how the objective function value is affected by that move. Afterwards a threshold value is used to determine if that move should be accepted or not. For maximization problems, the threshold value could simply be that the delta value should be above 0 in order for the move to be accepted. A critical function in a LS algorithm is thus the function that evaluates a move, as this is something that is done many times throughout the solution search. I found it to be highly inefficient to work with a function that evaluates only a single move. As the solution improves, more and more moves need to be tried before finding an improvement and a function that only evaluate a single move is too slow for this. Also, I found that by evaluating several moves, in a structured way, I found several tricks that speed up the evaluation. \\

The way I evaluate moves is by selecting a neuron in the neural network with weights going into it, so I do not select neurons from the input layer. For this neuron I evaluate the possible moves for the weights going into it. This is done simultaneously, but still independently in the way that I am testing what would happen if a single weight changes value, not what would happen if all of them change values at once. After this evaluation, a sequence of moves will be returned, each with their own delta value, that I can then use to determine what move to take. I will now introduce the most important aspects of this and will with an example show how it works. 

\subsubsection{Critical Samples}
Recall the binary activation function defined in (\ref{act}). The input to that function is the preactivation values, $s_v^{il}$ and in a LS context where only one-exchange moves are considered, there are values of these preactivation values such that the output of (\ref{act}) is unchanged regardless which of the weights going into neuron $v$ at layer $l$ change values. The weights can only take on values -1, 1 (0 as well for a TNN), so they can only change their value by either -2 or 2 in a BNN where -1 and 1 are also possibilities in a TNN. Thus, if we denote the maximum output value, or activation value, (in absolute value) from the previous layer by $u^{l-1}_{\max}$, then we can define the set of critical samples for a neuron, which are those samples that can change activation value by making a single move for one the weights going into that neuron, by:
\begin{align}
    \label{critical} \text{critical }^l_ v = \{ i \in \{1, \ldots, T\}: s_v^{il} \geq -2 \cdot u^{l-1}_{\max} \wedge s_v^{il} < 2 \cdot u^{l-1}_{\max} \}
\end{align}
I use a general notation to both describe the case for the first hidden layer, where the output from the previous layer are dependent on the input, but from the second hidden layer, it is possible to skip the $u^{l-1}_{\max}$ term, as this is equal to 1 as a result of the binary activation function. The purpose of these 'critical' samples, is that it greatly reduces the number of instances for which the delta evaluation need to be evaluated for. This trick is only used for the hidden layers, as the same does not apply for the output layer. 

\subsubsection{Forward Propagation}
The next trick that is useful when evaluating moves for a neuron is that it can be used to reduce the number of forward propagations. This logic applies both for a hidden layer and the output layer. It is a bit more complicated for the output layer in the TNN case, but nevertheless it is very useful. After having found the critical samples for a neuron (in a hidden layer), we need to evaluate what would happen if a move was applied to the weights going into the neuron. A naive way to do this would be to apply a forward propagation for all of the weights on the critical samples to see what would happen with the objective function value. But the binary activation function gives a way to do this more efficiently. The only way that something changes for a specific instance is if the activation of that instance changes, in which case we know that it changes it value by -2 or 2, dependent on the value of it before. Thus, we can simulate what would happen if the activations change for all the critical samples and find their 'delta changes' by doing a single forward propagation. Afterwards we can, for each weight we want to calculate the effect of a move for, find out if making the move would change the activations and then the 'delta changes' can be looked up instead of calculating it again. \\

\noindent For the output layer, a similar logic applies. Again a neuron is selected, but this time the critical set is not relevant, as the binary activation function is not used here. But again, for a BNN exactly one of two things happen (assuming the presence of at least one hidden layer): either the preactivation value increases by 2 or it decreases by 2. This comes from the fact that the output from the previous layer is either -1 or 1 and the value of the weight changes by either -2 or 2. Thus, it is again possible to find the 'delta changes' for the samples by simulating what would happen if their values increase by 2 and what would happen if they decrease by 2. Afterwards, for each single weight it can quickly be determined which of the cases a sample is in and the effect can be looked up in the 'delta changes'-tables. This greatly reduces the number of times the function calculating the objective function value needs to be called. For a TNN, it is almost the same, except 4 'delta changes'-tables are needed as the values can also increase and decrease by 1 in this case. 

\subsubsection{Example of Delta Evaluation}
To illustrate how the mechanisms described above works, I have constructed a BNN with the following structure $[784, 4, 4, 4, 10]$. I am training it on a balanced training set of 10 instances from the MNIST dataset. Initially I initialize a solution by randomly assigning values to all the weights and I am now looking for what move to make in order to improve the current solution. Suppose now that I look at the fourth neuron in the second hidden layer. The preactivations are given in the vector:

\begin{align*}
    S_4^2 = 
    \begin{bmatrix}
        0 & 4 & 4 & 2 & 2 & 2 & 4 & 2 & -2 & 2
    \end{bmatrix}
\end{align*}
Each element corresponds to a different training sample. Since we are in the second hidden layer, $u^1_{\max} = 1$ and thus $\text{critical }_4 ^2 = \{1, 9\}$, as it is only the instances with 0 and -2 as preactivation values that can change activation. Thus, for the remainder of the delta evaluation process for this neuron, it is only necessary to look at these two instances. The first thing needed to do is to simulate what would happen if their activations change. Thus, I start by calculating the effect of this by propagathing through the network. During the process, I always try to do as little work as possible meaning that whenever I can benefit from using the values already stored I do so. The preactivation values for the next layer for these two instances is currently given by:

\begin{align*}
    S^3 = 
    \begin{bmatrix}
        -2 & 0 & -2 & 0 \\
        -2 & 0 & -2 & 0 
    \end{bmatrix}
\end{align*}
where each row corresponds to an instance and each column to a neuron at the next layer. The weight vector going into the next layer from the fourth neuron in the second layer is given by:
\begin{align*}
    W = 
    \begin{bmatrix}
        -1 & 1 & 1 & 1
    \end{bmatrix}
\end{align*}
Looking at the critical instances it is easy to recognize, that if they changed sign and thus activation, then the first would decrease from 1 to -1 and the second would increase from -1 to 1. Thus, we have:
\begin{align*}
    \hat{S}^3 = S^3 + 
    \begin{bmatrix}
        -2 \\
        2
    \end{bmatrix}
    \circ 
    \begin{bmatrix}
        -1 & 1 & 1 & 1 
    \end{bmatrix}
    = 
    \begin{bmatrix}
        0 & -2 & -4 & -2 \\
        -4 & 2 & 0 & 2 
    \end{bmatrix}
\end{align*}
Here $\circ$ is elementwise mulplication. \\
\noindent Until now it has been possible to calculate what is happening by looking at the preactivation values stored in memory and updating them. This is efficiently to do for the neuron that is being evaluated and for the next layer, but afterwards it makes more sense to forget what is in memory and instead finish the forward propagation with the temporary matrices. To finish the forward propagation one need to apply the activation function, (\ref{act}), to $\hat{S}^3$, yielding the result: 
\begin{align*}
    \hat{U}^3 = 
    \begin{bmatrix}
        1 & -1 & -1 & -1 \\
        -1 & 1 & 1 & 1 
    \end{bmatrix}
\end{align*}
The last step is to multiply with the last weight matrix to obtain the preactivation values for the last layer. Here I will only give the result:
\begin{align*}
    \hat{S^4} = \hat{U^3}W^4 = 
    \begin{bmatrix}
        2 & 0 & 0 & -4 & 4 & 2 & -2 & 2 & 0 & 2 \\
        -2 & 0 & 0 & 4 & -4 & -2 & 2 & -2 & 0 & -2
    \end{bmatrix}
\end{align*}
Compare this to the current preactivation values, which is determined after the random initialization:
\begin{align*}
    S^4 = 
    \begin{bmatrix}
        -4 & -2 & 2 & 2 & -2 & 0 & 0 & 0 & -2 & -4 \\
        -4 & -2 & 2 & 2 & -2 & 0 & 0 & 0 & -2 & -4
    \end{bmatrix}
\end{align*}
Since the correct labels for these two instances are 4 and 9 respectively (using 0-indexing), the objective vector, using the integer objective function, is initially given by:
\begin{align*}
    O = 
    \begin{bmatrix}
        -2 - 2 \\
        -2 - 4 
    \end{bmatrix}
    = 
    \begin{bmatrix}
        -4 \\
        -6 
    \end{bmatrix}
\end{align*}
whereas using $\hat{S}^4$, the result is:
\begin{align*}
    \hat{O} = 
    \begin{bmatrix}
        4 - 2 \\
        -4 -2 
    \end{bmatrix}
    = 
    \begin{bmatrix}
        2 \\
        -6 
    \end{bmatrix}
\end{align*}
This gives a vector of delta changes:
\begin{align*}
    D = \hat{O} - O = 
    \begin{bmatrix}
        2 - (-4) \\
        -6 - (-6) 
    \end{bmatrix}
    = 
    \begin{bmatrix}
        6 \\ 
        0 
    \end{bmatrix}
\end{align*}
This means that we have found out that if the activations change for the two critical instances, then this has a positive effect for one of them and zero effect for the other. The last thing we need to do is to find out which, if any, of the four weights going into the neuron can make the activations change. The current weights going into this neuron has the values: 
\begin{align*}
    W_4^2 = 
    \begin{bmatrix}
        -1 & 1 & 1 & 1
    \end{bmatrix}
\end{align*}
and the output of the previous layer for the two critical instances are:
\begin{align*}
    U^1 = 
    \begin{bmatrix}
        1 & -1 & 1 & 1 \\
        1 & -1 & -1 & 1
    \end{bmatrix}
\end{align*}
What I then do is that I create four copies of the preactivation values for the critical instances, one for each weight going into the neuron. Afterwards I simulate what would happen if the weights are 'flipped', take the activations of these simulated preactivation values and compare them to the current activations. This gives me a matrix where each row represents an instance and the column represents a weight. The elements indicate whether the activation has changed or not, meaning that to get the effect of a weight we can columnwise take the inner product between the column and the delta changes vector, $D$, found earlier. I start by finding the simulated preactivation values:
\begin{align*}
    \hat{S}^2_4 = 
    \begin{bmatrix}
        0 & 0 & 0 & 0 \\
        -2 & -2 & -2 & -2
    \end{bmatrix}
    + 
    \begin{bmatrix}
        1 & -1 & 1 & 1 \\
        1 & -1 & -1 & 1
    \end{bmatrix}
    \circ 
    \begin{bmatrix}
        2 & -2 & -2 & -2
    \end{bmatrix}
    =
    \begin{bmatrix}
        2 & 2 & -2 & -2 \\
        0 & 0 & 0 & -4
    \end{bmatrix}
\end{align*}
I then need to find out where there is a change in activation values compared to the current solution. Clearly, since the current preactivation value for the first instance is 0, this activation is currently 1, and for the second it is -1 as the preactivation value is -2. Thus, it gives the following 'changes' matrix, where 1 indicate that the activation has changed and 0 indicate that it has not. 
\begin{align*}
    \text{changes} = 
    \begin{bmatrix}
        0 & 0 & 1 & 1 \\
        1 & 1 & 1 & 0
    \end{bmatrix}
\end{align*}
As an example this shows that for the weight indexed by the third column, flipping that value from 1 to -1, will change the activation of both the critical instances. The last remaining thing to do is to, columnwise found the effect of flipping each weight. This gives the following delta values:
\begin{align*}
    DW = 
    \begin{bmatrix}
        0 \cdot 0 + 1 \cdot 0 & 0 \cdot 0 + 1 \cdot 0 & 1 \cdot 6 + 1 \cdot 0 & 1 \cdot 6 + 0 \cdot 0
    \end{bmatrix}
    = 
    \begin{bmatrix}
        0 & 0 & 6 & 6
    \end{bmatrix}
\end{align*}
Thus, an improvement of 6 in terms of the integer objective function can be found by flipping the third or fourth weight going into this neuron. 





\subsection{Solution Improvement}
Recall, that the goal of the classification problem is to be able to classify test samples that are not seen during training. During training, the model is trained on batches of training samples, and an objective function is used to evaluate the quality of the model. I will test the effect of using a single batch compared to using several, and a key ingredient in most of my algorithms is the iterated improvement algorithm. In words, this algorithm takes a solution and searches for improvements. I do this by iterating through the neurons in the network as described earlier. For each neuron, it takes the best move found and checks whether this is an improvement, and if it is, the move is committed and the current solution is updated. The algorithm only stops when the time limit has been reached or the solution has reached a local optima, which is when all the neurons have been checked without finding an improvement. In practice, it is quite fast to end up in a local optima for a single batch. The pseudocode for this algorithm can be seen in Algorithm \ref{iterated_improvement}. \\

\begin{algorithm}[!tb]
    \caption{Pseudocode for Iterated Improvement} \label{iterated_improvement}
    \begin{algorithmic}[1] 
        \State \textbf{Input:}
        \State \hspace{\algorithmicindent} Initial solution $currentSolution$
        \State \hspace{\algorithmicindent} Time limit $timeLimit$
        \State $startTime \gets$ current time
        \While{current time $-$ $startTime < timeLimit$}  
            \State Shuffle the nodes of $currentSolution$
            \State $improvementFlag \gets \text{False}$
            \For{each $node$ in $currentSolution$}
                \State $bestMove \gets$ findBestMove($currentSolution$, $node$)
                \If{delta value of  $bestMove > 0 $}
                    \State $currentSolution \gets $ applyMove($currentSolution$, $bestMove$)
                    \State $improvementFlag \gets \text{True}$
                \EndIf
                \If{current time $-$ $startTime \geq timeLimit$}  
                    \State \textbf{return} $currentSolution$
                \EndIf
            \EndFor
            \If{$improvementFlag = \text{False}$}  
                \State \textbf{return} $currentSolution$
            \EndIf
        \EndWhile
        \State \textbf{return} $currentSolution$
    \end{algorithmic}
\end{algorithm}

\noindent Using only Algorithm \ref{iterated_improvement} will typically not yield good results. While training a neural network, many local optima are usually visited, and traditional neural network training has several techniques to escape from a local optima. I will use a simple optimization metaheuristic, iterated local search (ILS), to escape from the local optima. This metaheuristic works by taking the current solution, which is a locally optimal solution, and use it to get another solution, which is no longer locally optimal. Afterward, iterated improvement will be applied to the new solution until it arrives at a local optima and the process repeats itself until the solution has converged or the time limit has been reached. The strategy used to get another solution from the current solution is called perturbation. In the context of training BNNs and TNNs, it works by taking a solution and randomly changing the values of a number of the weights. The important parameter here is how many weights to change values of. If too few weights are changed, there is a high chance that the solution falls back to the same local optima it came from. On the other hand, if too many weights are changed, it could be the same as using random restart, which could mean that the solution quality does not improve much. The pseudocode is given in Algorithm \ref{ils}. \\

\noindent Another problematic aspect of this algorithm is that it is more difficult to determine when the model has converged. Instead of a convergence criterion, I use a time limit, such that the runtime is controlled. A convergence criterion could be to set a maximum number of perturbations allowed without ending up in a local optima that is better than the best seen so far. 

\begin{algorithm}[!tb]
    \caption{Pseudocode for Iterated Local Search} \label{ils}
    \begin{algorithmic}[1]
        \State \textbf{Input:}
        \State \hspace{\algorithmicindent} Initial solution $currentSolution$
        \State \hspace{\algorithmicindent} Time limit $timeLimit$
        \State \hspace{\algorithmicindent} Perturbation size $ps$
        \State $startTime \gets$ current time 
        \State $bestSolution \gets currentSolution$ 
        \While{current time $-$ $startTime < timeLimit$}
        \State $currentSolution \gets$ IteratedImprovement($currentSolution$,
        \Statex \hspace{\algorithmicindent} \hspace{\algorithmicindent}$timeLimit$ - (current time - $startTime)$)
        \If{$currentSolution$ $>$ $bestSolution$}
            \State $bestSolution \gets currentSolution$ 
        \ElsIf{$currentSolution < bestSolution$}
            \State $currentSolution \gets bestSolution$ 
        \EndIf
        \State $currentSolution \gets$ Perturb($currentSolution$, $ps$)
        \EndWhile
        \If{$currentSolution > bestSolution$}
            \State \textbf{return} $currentSolution$
        \Else 
            \State \textbf{return} $bestSolution$
        \EndIf
    \end{algorithmic}
\end{algorithm}

\subsection{Multiple Batch Training}

One of the ways I try to make sure that the models trained generalize well is to make use of more data and train the network on several batches. It is the same solution that is being trained on all of the batches, in the sense that the solution after training one batch is the starting solution to the next batch. This gives rise to a couple of problems. The first one is, how do we know what the 'best' solution is? If the solution returned is the one after the last batch, then it might be that it is heavily influenced by the last batch and as such it would have been better to use one of the solutions earlier on. Another consideration is how to make sure that the solution does not forget what it has learned from the previous batch. My attempt to solve these problems is to use an early stopping technique, where I after every $k$ batches get the validation accuracy on the validation dataset and after the complete training process I return the solution with the best validation accuracy. Of course, this introduces another parameter, $k$. Ideally one could set $k=1$ and get the validation accuracy for all the solutions, but this might be too costly as getting the validation accuracy involves evaluating the solution on a large dataset. \\

\noindent One of the concerns by training a model on several batches of data is that the model overfits on the batch it is currently trained on and forgets what it learned from the previous batch. For this reason, I investigate whether a sporadic local search approach helps the model to generalize better. This works by setting a parameter $bp$, which is a parameter in the interval $[0,1]$, which is given to the Bernoulli distribution, which then returns a '1' with probability $bp$ and a 0 otherwise. Before training on a batch, each weight samples a value from this distribution, and if the value is 1, then the weight is a part of the search for that batch and otherwise the value is kept fixed for that particular batch. For each new batch, new weights are selected to be part of the search. The hope is that this helps the model to avoid getting too focused on the current batch and reduce the overfitting gap. \\

\noindent The pseudocode for this algorithm is given in Algorithm \ref{multiple_batches}. In line 15 I give an additional parameter compared to the iterated improvement pseudocode in Algorithm \ref{iterated_improvement}. In practice it works a bit different, but this is just to underline that not all weights are part of the search. Notice, that this could easily be adjusted to using Algorithm \ref{ils} instead in line 15. The only difference would be that the algorithm needs to take a perturbation size as input and that each ILS phase are given a fixed time limit instead as it would continue indefinetely otherwise. 

\begin{algorithm}[H]
    \fontsize{12pt}{12pt}\selectfont
    \caption{Pseudocode for Multiple Batch Iterated Improvement} \label{multiple_batches}
    \begin{algorithmic}[1]
        \State \textbf{Input:}
        \State \hspace{\algorithmicindent} Time limit $timeLimit$
        \State \hspace{\algorithmicindent} Set of batches $batches$
        \State \hspace{\algorithmicindent} Interval to early stopping $k$
        \State \hspace{\algorithmicindent} Number of epochs $epochs$
        \State \hspace{\algorithmicindent} Bernoulli parameter $bp$ 
        \State $startTime \gets$ current time 
        \State $currentSolution \gets$ random solution
        \State $bestSolution \gets currentSolution$
        \State $bestValidationAccuracy \gets$ ValidationAccuracy($currentSolution$)
        \State $counter \gets 0$
        \While{current time $-$ $startTime < timeLimit$} 
            \For{$epoch$ in range($epochs$)}
                \For{$batch$ in $batches$}
                    \State $searchWeights \gets$ SelectWeights($bp$)
                    \State $currentSolution \gets$ IteratedImprovement($currentSolution$,
                    \Statex \hspace{\algorithmicindent} \hspace{\algorithmicindent}$timeLimit$ - (current time - $startTime)$, $searchWeights$)
                    \If{$counter$ modulo $k$ $ = 0$}
                        \State $ValidationAccuracy \gets$ ValidationAccuracy($currentSolution$)
                        \If{$ValidationAccuracy > bestValidationAccuracy$}
                            \State $bestSolution \gets currentSolution$ 
                            \State $bestValidationAccuracy \gets ValidationAccuracy$ 
                        \EndIf
                    \EndIf
                    \State $counter \gets counter + 1$
                    \If{current time $-$ $startTime \geq timeLimit$}  % Time check within loop
                        \State \textbf{return} $bestSolution$
                    \EndIf
                \EndFor
                \State $batches \gets$ ResampleBatches 
            \EndFor
        \EndWhile
        \State \textbf{return} $bestSolution$
    \end{algorithmic}
\end{algorithm}

\noindent So far I have looked at algorithms making moves based on one batch and tried to make sure that the model does not forget what it has learnt from other batches by only looking at a subset of the weights. An alternative method is to make less moves, but making sure that the moves generalize better. One possibility for this is to sum up delta values for the moves across several batches and only after a certain number of batches, some moves are committed. This should make sure that the moves committed benefit not only a single batch of samples, but multiple batches. An important decision to make is when to make updates. The more batches seen before making updates, the more confident will the algorithm be that the moves generalize well, but it will also be slower. For this reason I add another aspect to the algorithm, such that it initially make updates based on very few batches and later on it makes updates on more batches. The parameters for this are given in line 5-7 in Algorithm \ref{multiple_batches_v2}, which shows the pseudocode for this algorithm. \\

A problematic aspect of this algorithm is that the delta values are calculated under the assumption that a single move is taken, but here, to speed up the process, I take many moves at once. However, to avoid the moves interfering too much with each other I only take one move per neuron. A different problem is again the convergence problem, where I, to overcome this problem, use a time limit. For this algorithm I do not use early stopping as the use of several batches to make updates should make sure that it generalizes better compared to the version introduced earlier. 


\begin{algorithm}[H]
    % \footnotesize
    \fontsize{12pt}{12pt}\selectfont
    \caption{Pseudocode for Multiple Batch Aggregation Algorithm} \label{multiple_batches_v2}
    \begin{algorithmic}[1]
        \State \textbf{Input:}
        \State \hspace{\algorithmicindent} Time limit $timeLimit$
        \State \hspace{\algorithmicindent} Set of batches $batches$
        \State \hspace{\algorithmicindent} Bernoulli parameter $bp$ 
        \State \hspace{\algorithmicindent} How many batches before making updates in the beginning $updateStart$
        \State \hspace{\algorithmicindent} The maximum number of batches before making updates $updateEnd$
        \State \hspace{\algorithmicindent} How often to increase the update interval $updateIncrease$
        \State $startTime \gets$ current time 
        \State $currentSolution \gets$ random solution
        \State $updateInterval \gets$ $updateStart$
        \State $counter, \; updateCounter \gets 0, \; 0$    
        \State $searchWeights \gets$ SelectWeights($bp$)
        \State $deltaValues$ $\gets$ $0$
        \While{current time $-$ $startTime < timeLimit$}
            \For{$batch$ in $batches$}
                \State $updateCounter \gets updateCounter + 1 $
                \State $deltaValues \gets deltaValues$ + 
                \Statex \hspace{\algorithmicindent} \hspace{\algorithmicindent} CalculateDeltaValues($currentSolution$, $batch$, $searchWeights$)
                \If{$updateCounter = updateInterval$}
                    \For{each $node$ in $currentSolution$}
                        \State $bestMove \gets$ findBestMove($deltaValues$, $node$)
                        \If{delta value of  $bestMove > 0$}
                            \State applyMove($currentSolution$, $bestMove$)
                        \EndIf
                    \EndFor
                    \State $updateCounter \gets 0$ 
                    \State $deltaValues$ $\gets$ $0$
                    \If{$counter$ modulo $updateIncrease$ $ = 0$}
                        \State $updateInterval \gets updateInterval + 1$ 
                    \EndIf
                \EndIf
                \State $counter \gets counter + 1$
            \EndFor
            \State $batches \gets$ ResampleBatches 
        \EndWhile 
        \State \textbf{return} $bestSolution$
    \end{algorithmic}
\end{algorithm}





\subsection{Code Organization}

One of the major challenges of this thesis has been to develop the framework for training BNNs and TNNs. The framework needed to be quite flexible so that it allows both BNNs and TNNs, but also ensures that all of the algorithms and experiments can be tested within the same framework. All of the source code for this thesis can be found at: https://github.com/mbruh19/Master-Thesis. The implementation is done in Python. To train a BNN or TNN, 'main.py' needs to be called with the right parameters. From this file, everything else runs automatically. As a starting point, it initializes the 'Reader' class and loads the training, validation, and testing sets by calling the load data function. The current framework supports loading the MNIST, Fashion-MNIST (FMNIST) and the Adult dataset. To load from other datasets, the necessary function needs to be implemented in the Reader class. Next, the 'Instance' class is called, which takes all the settings of the experiment as input as well as the three datasets. The Instance class has the very important 'loader' function, needed to iterate through batches from one of the datasets. \\

\noindent Having this, one of the algorithms presented is called. When an algorithm is called, it initially starts by loading a batch and creating a 'Solution' object. This object is the core of the implementation. The most important attributes of this object are the $W$, $S$ and $U$ matrices introduced earlier as well as a vector, $O$ denoting the contribution of each sample to the objective function. It also has an attribute of the nodes in the network, that can be iterated through. The $S$, $U$ and $O$ attributes are dependent on the current batch, and as a result, the Solution object has a function to change the batch, which includes re-evaluating these attributes. The object also has functions to initialize a random solution and copy the weights. More importantly, it is also here the function to commit a move is located, where the necessary updates are done. The remaining functions are those to perturb the solution, used in ILS, and a function to select search weights, used in sporadic local search. \\

\noindent The algorithm then moves on and calls one of the 'solvers' implemented. There are two 'solvers' implemented, iterated improvement, which follows Algorithm \ref{iterated_improvement} or iterated local search, described in Algorithm \ref{ils}, which uses Algorithm \ref{iterated_improvement}. In the iterated improvement algorithm, the 'Delta Manager' object is called. This object has a single function, delta calculation, which either uses a delta function for BNN or TNN. This function takes the current solution and a node as input. For this node, all the weights going into this node, which are part of the search, are tested to see what the delta value is if the weight changes value. This follows the procedure described in section 5.4. The delta calculation function returns a sequence of moves back to the iterated improvement algorithm, which takes the best of the moves and decides whether to commit the move or not. If the algorithm uses multiple batch training, it loads a new batch, changes the batch on the solution object. and the process repeats itself until the time limit has been reached. \\

% Here I plan to include the following:

% \begin{itemize}
%     \item Intro to Local Search
%     \item Neighborhood
%     \item Local Search Algorithms - iterative improvement, first improvement, best improvement, random improvement
%     \item Local Search Metaheuristics - iterated local search 

% \end{itemize}